Consider a generic electronic system, which is fabricated on a silicon wafer hosting $\ndies$ dies.
The system depends on a process parameter $\u$, which we are interested in studying and shall refer to as the quantity of interest (\qoi).
Due to the presence of process variation, the value of $\u$ deviates from the nominal one, and this deviation can be different at different locations on the wafer.
The \qoi\ is assumed to be expensive/impractical for direct measurements.

The goal of this work is to develop a statistical framework targeted at the identification of the on-wafer distribution of $\u$ with the following properties: (a) low measurement costs; (b) high computational speed; (c) robustness to the measurement noise; (d) ability to accommodate prior knowledge on $\u$; and (e) ability to assess the trustworthiness of the collected data and corresponding predictions.

In order to achieve the established goal, we propose the use of indirect measurements.
Specifically, instead of $\u$, we measure an auxiliary parameter $\q$, which we shall refer to as the quantity of measurement (\qom).
The observations of $\q$ are then processed via Bayesian inference in order to derive the distribution of the \qoi, $\u$.
The \qom\ is chosen such that: (a) $\q$ is convenient and cheap to be tracked; (b) $\q$ depends on $\u$, which is signified by $\q = \oBB{\u}$; and (c) there is a way to compute $\q$ for a given $\u$.
The last means that $\oBB$ should be known; however, it does not have to be explicitly given: our framework treats $\oBB$ as a ``black box.''
For example, $\oBB$ can be a piece of code or an output of an adequate simulator.

As the first step, the user of the proposed framework is supposed to harvest a set of observations of $\q$ at several locations on the wafer (recall \sref{motivation}).
Without loss of generality, we shall adhere to the following convention.
One die corresponds to one potential measurement site, and $\nrdies \ll \ndies$ denotes the number of those sites that have been selected for measurements.
Each site comprises $\nprocs$ measurement points, and each point contains $\nsteps$ data instances.
For example, in \sref{motivation}, each observation was an $\nprocs \times \nsteps$ matrix capturing temperature of $\nprocs$ processing elements for $\nsteps$ moments of time.
Denote by $\QData = \{ \q_i^\meas \}_{i = 1}^\nrdies$ the collected data set where $\q_i^\meas \in \real^{\nprocs \times \nsteps}$ stands for one observation (one site) of the \qom.
It is implied that the placement of each selected site is recorded along with $\QData$.

Note that, if $\oBB$ is the identity function, \ie, $\q \equiv \u$, the proposed technique will primarily focus on the reconstruction of any missing observations (defined in \sref{model-order-reduction}) in $\QData$.
From this standpoint, our approach is a generalization of those developed in \cite{zhang2010, reda2009}.

For convenience, we denote by $\specification$ all the information relevant to the production and measurement processes including: (a) the layout of the wafer and (b) the floorplan of a die on the wafer.
