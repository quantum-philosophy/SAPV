Since, at each point of the continuum of spatial locations on the wafer, $\U$ can potentially take a different value, $\U$ is infinite-dimensional. We model $\U$ as a stochastic process $\U: \outcomes \times \domain \to \real$ in $\H$ defined over a spacial domain $\domain$ corresponding to the wafer. Denote the forward model developed in \sref{power-model} and \sref{thermal-model} by
\begin{equation} \elabel{model}
  \mT(\r) = \model{\u, \r}
\end{equation}
which, for an outcome $\u: \domain \to \real$ of the process $\U$, transforms the test dynamic power profile $\profilePdyn$ into the corresponding temperature profile $\profileT$ computed at location $\r \in \domain$. Note, $\model$ also performs thinning of the output temperature to match the partition $\partition{\mT}$ of $\data$.

\subsubsection{Additive noise}
Due to the imperfection of the modeling and measurement processes, a temperature profile $\mT^{(i)}_\meas$ from the data set $\data$ is assumed to deviate from the model prediction in \eref{model}. To account for this, $\mT^{(i)}_\meas$ is modeled as
\[
  \mT^{(i)}_\meas = \mT(\r_i) + \mnoise^{(i)} = \model{\u, \r_i} + \mnoise^{(i)}
\]
where $\mnoise^{(i)} = (\noise^{(i)}_{jk}) \in \real^{\ncores \times \nsteps}$ represents a matrix of an additive noise. The noise is typically assumed to be a white Gaussian noise and to be independent from the parametrization $\U$ \cite{marzouk2007, el-moselhy2012}. Therefore, $\noise^{(i)}_{jk}$ is modeled as
\[
  \noise^{(i)}_{jk} | \param^2_\noise \sim \gaussian{0}{\param^2_\noise}
\]
where $\param^2_\noise$ is a parameter defining the variance of the noise; imposing no loss of generality, the noise is assumed to have the same magnitude for all measurements.

\subsubsection{The likelihood function}
The noise above yields the following likelihood $\f{\data | \u}$ of the data $\data$, which, intuitively speaking, gives the probability of observing $\data$:
\begin{align}
  \f{\data | \u} & = \prod_{i = 1}^{\ndata} \f{\mT^{(i)}_\meas| \u, \r_i} \nonumber \\
  & \propto \frac{1}{\param_\noise^\ndcs} \exp\left( -\frac{1}{2 \param^2_\noise} \sum_{i = 1}^\ndata \fnorm{\mnoise^{(i)}}^2 \right) \elabel{likelihood}
\end{align}
where $\ndcs = \ndata \ncores \nsteps$, $\fnorm{\cdot}$ is the Frobenius norm, and
\[
  \mnoise^{(i)} = \mT^{(i)}_{\meas} - \mT(\r_i) = \mT^{(i)}_{\meas} - \model{\u, \r_i}.
\]

\subsubsection{Stochastic process}
We need to identify $\U$ in order to specify the prior in \eref{bayes}. Uncertainties due to process variation are known to be well approximated using Gaussian distributions \cite{srivastava2010}; therefore, $\U$ is assumed to be a Gaussian process \cite{rasmussen2006}:
\[
  \U \sim \gaussian{\fMean{\cdot}}{\fCov{\cdot, \cdot}}
\]
where $\fMean{\r} := \E{\U(\r)}$ and $\fCov{\r, \r'}$, $\forall \r, \r' \in \domain$, are the mean and covariance functions of $\U$. Due to the particularities of the manufacturing process, such uncertainties often bear radial structures \cite{cheng2011}; thus, we assume that $\r$ represents the Euclidean distance from the center of the wafer to the center of the die, and the covariance function of $\U$ is exponential \cite{maitre2010}:
\begin{equation} \elabel{covariance-function}
  \fCov{\r, \r'} = \param_\U^2 \exp\left(-\frac{\abs{\r - \r'}}{\ell}\right)
\end{equation}
where $\param_\U^2$ and $\ell > 0$ parametrize the function and stand for the variance and correlation length, respectively.

\subsubsection{Model reduction}
The infinite-dimensional object $\U$ is reduced to the finite-dimensional case using the KL expansion introduced in \sref{kl-expansion}:
\begin{equation} \elabel{kl-approximation}
  \U(\r) \approx \klapprox{\U}{\vZ, \r} := \fMean + \sum_{i = 1}^\nvars \sqrt{\lambda_i} \: \klb_i(\r) \: \Z_i
\end{equation}
where $\{ (\lambda_i, \klb_i) \}$ are the first $\nvars$ eigenpairs of $\fCov{\cdot, \cdot}$, $\Z_i$ are standard Gaussians, and the mean function $\fMean{\cdot}$ is modeled as a constant equal to the nominal value $\fMean$ of $\U$. To summarize, we have replaced the parametrization $\U$, given as a stochastic process, with the parametrization $\vZ = (\Z_i) \in \real^\nvars$, given as an $\nvars$-dimensional vector of \rvs. Assuming $\ell$ and $\fMean$ be given, we define a new parametrization of the model as follows:
\[
  \Param = \left\{ \Z_1, \dotsc, \Z_\nvars, \param^2_\U, \param^2_\noise \right\},
\]

\subsubsection{The prior}
The parameters in $\Param$ are assumed to have the following priors \cite{marzouk2009}:
\begin{align*}
  & \Z_i | \param^2_\U \sim \gaussian{0}{\param^2_\U}, \\
  & \param^2_\U \sim \sichisquared{\nu_\U}{\tau^2_\U}, \\
  & \param^2_\noise \sim \sichisquared{\nu_\noise}{\tau^2_\noise}.
\end{align*}
The prior for $\Z_i$ is due to the definition of a KL expansion for Gaussian processes (see \sref{kl-expansion}) and due to the multiplier $\param^2_\U$ of the covariance function. The last two priors, the scaled inverse chi-squared distributions, are a common, in Bayesian settings, choice for variance of a Gaussian model \cite{gelman2004}. The parameters $\nu_\noise$, $\nu_\U$, $\tau^2$, and $\tau^2_\U$ should be set according to the prior beliefs. In our experiments, we fix $\nu_\noise = \nu_\U = 2$ and $\tau^2_\noise = \tau^2_\U = 1$, which results in a heavy-tailed distribution allowing one for having a rather limited prior knowledge.

% Now, let $\vr^* = (\r^*_i) \in \real^n$ be a vector of spatial locations of interest. Then the prior over these locations is
% \begin{equation} \elabel{prior}
%   \f{\u | \vr^*} = \left(|2 \pi \mCov|\right)^{-\frac{1}{2}} \exp\left(-\frac{1}{2} (\vu - \vfMean)^T \mCov^{-1} (\vu - \vfMean) \right)
% \end{equation}
% where $\vu = \u(\vr^*) := (u(\r^*_i)) \in \real^n$, $\vfMean = \fMean{\vr^*} := (\fMean{\r^*_i}) \in \real^n$, $\mCov = (\fCov{\r^*_i, \r^*_j}) \in \real^{n \times n}$, and $|\cdot|$, applied to a matrix, denotes the matrix determinant. Equivalently,
% \[
%   \u | \vr^* \sim \gaussian{\vfMean}{\mCov}
% \]
% where $\u | \vr^*$ stands for $\u(\vr^*)$.
%
% Due to the forward model $\model$ involved in \eref{likelihood}, there is no convenient expression for the posterior density $\f{\u | \data}$, which would allow one for a straightforward sampling from the posterior. To overcome the difficulty, we utilize the Metropolis-Hastings algorithm \cite{gelman2004}. To this end, we combine \eref{likelihood} with \eref{prior} and compute, up to a constant summand, the logarithm of the posterior:
% \begin{align*}
%   & \log \f{\u | \data, \vr^*} = \log \f{\data | \u} + \log \f{\u | \vr} + c_1 \\
%   & = -\frac{1}{2 \param^2_\noise} \sum_{i = 1}^\ndata \fnorm{\mnoise^{(i)}}^2 -\frac{1}{2} (\vu - \vfMean)^T \mCov^{-1} (\vu - \vfMean) + c_2 \\
%   & = - A(\u | \data) - B(\u | \vr^*) + c_2
% \end{align*}
% where $c_1$ and $c_2$ are constants with respect to $\u$, which are irrelevant for our purposes, and
% \begin{align*}
%   & A(\u | \data ) = \frac{1}{2 \param^2_\noise} \sum_{i = 1}^\ndata \fnorm{\mT^{(i)}_{\meas} - \model{\r_i | \u}}^2, \\
%   & B(\u | \vr^*) = \frac{1}{2} (\vu - \vfMean)^T \mCov^{-1} (\vu - \vfMean).
% \end{align*}
% It is worth being noted that $A(\u | \data)$ poses a significant computational challenge as it requires $\ndata$ evaluations of $\model$ for each sample of $\U$.
