Process variation constitutes one of the major concerns of multiprocessor system designs \cite{srivastava2010}. A crucial implication of process variation is that it renders the key parameters of a technological process, \eg, the effective channel length, gate oxide thickness, threshold voltage, \etc, as random variables.
Therefore, a purely deterministic workload applied to two ``identical'' dies can potentially lead to two different power and, thus, temperature profiles since such profiles intrinsically depend on these stochastic parameters. Consequently, process variation leads to performance degradation in the best case and to burnt silicon in the worst.
Under these circumstances, uncertainty quantification (UQ) has evolved into one of the indispensable assets of successful multiprocessor system design workflows, \ie, those that provide high guaranties on the efficiency and robustness of their products.

A important target of UQ is the characterization of the on-wafer distribution of a quantity of interest (QoI), deteriorated by process variation, which is typically based on a data set of measurements. The problem belongs to the class of inverse UQ problems as the QoI is an input and the measurements are an output; this is opposed to direct problems wherein one studies outputs based on the knowledge of inputs (see, \eg, \cite{juan2011, juan2012}).
Such an inverse problem is addressed in this paper: our goal is to quantify the distribution of the key process parameters mentioned earlier, and we approach this goal by measuring temperature and analyzing it using Bayesian inference \cite{gelman2004}.

Bayesian inference is utilized in \cite{zhang2010} to identify an optimal set of spatial locations on the wafer, in which the QoI, such as the effective channel length, should be measured in order to characterize it with the maximal accuracy.
In \cite{paek2012}, the authors consider an inverse problem focused on the inference of the power dissipation based on transient temperature profiles using Markov random fields.
Another temperature-based characterization of power is developed in \cite{mesa-martinez2007} wherein a genetic algorithm is employed for reconstruction of the power model.
It should be noted that the approach in \cite{zhang2010} requires test structures to be deployed onto each die on the wafer as it operates on direct observations of the QoI. This can be expensive and, thus, impractical to undertake. The techniques in \cite{paek2012} and \cite{mesa-martinez2007} solely focus on power and do not attempt to dive deeper and quantify the primary sources of uncertainty.

The contribution of our work is in the following. First of all, we propose a novel approach to UQ of process variation based on indirect, incomplete, and noisy measurements. Indirectness is the key ingredient as it allows for a significant decrease of the costs associated with the process variation identification. Second, we develop a solid framework around the proposed idea to make the method readily available for practical instantiations. Third, we present a thorough study of various aspects of the framework that are of a high practical interest.

The reminder of this paper is organized as follows. An application of our technique is described in \sref{motivation}. In \sref{problem-formulation}, we formulate the considered problem and state our solution. The proposed UQ framework is developed in \sref{proposed-framework}. Experimental results are reported in \sref{experimental-results}. \sref{conclusion} concludes the paper. The paper also contains a set of supplementary materials given in appendix, \aref{bayesian-inference} and \aref{kl-expansion}.
