Process variation constitutes one of the major concerns of multiprocessor system designs. The crucial implication of process variation is that it renders the key parameters of a technological process, \eg, the effective channel length, gate oxide thickness, threshold voltage, \etc, as random variables (\rvs). In particular, it implies that a purely deterministic workload applied to two ``identical'' dies can potentially lead to two different power and temperature profiles since such profiles directly depend on these stochastic parameters. Consequently, the phenomenon of process variation results in performance degradation and in burnt chips. In this essence, the concept of uncertainty quantification (UQ) has become central for the multiprocessor system design workflows that are concerned with efficiency and robustness. In this paper, we consider an inverse UQ problem targeted at the inference of the above-mentioned parameters, deteriorated by process variation, based on a data set of measurements, which we shall analyze using the ideas originating from Bayesian inference \cite{gelman2004}.

Bayesian inference is utilized in \cite{zhang2010} to identify an optimal set of spatial locations on the wafer, in which the quantity of interest (QoI), such as the effective channel length, should be measured in order to characterize it with the maximal accuracy. In \cite{paek2012}, the authors address the inference of the power dissipation based on transient temperature profiles using Markov random fields. Another temperature-based characterization of power is developed in \cite{mesa-martinez2007} wherein a genetic algorithm is employed for reconstruction of the power model. It should be noted that the approach in \cite{zhang2010} requires measurement structures to be deployed onto each die on the wafer as it operates on direct observations of the QoI. This can be expensive and, thus, impractical to undertake. The techniques in \cite{paek2012} and \cite{mesa-martinez2007} solely focus on power and do attempt to infer other parameters.

The contribution of our work is in the following. First of all, we propose a novel approach to UQ of process variation based on indirect, incomplete, and noisy measurements. Indirectness is the key ingredient as allows for a significant decrease of the costs associated with the process variation identification. Secondly, we develop a solid framework around the proposed idea to make the method readily available for practical implementations. Thirdly, we present a thorough study of various aspects of the framework that are of high practical interests.

The reminder of this paper is organized as follows. A motivational example is given in \sref{motivation}. In \sref{problem-formulation}, we formulate the objective of our study. Preliminary materials are given in \sref{preliminaries}. The proposed UQ framework is presented in \sref{proposed-framework}. Experimental results are reported in \sref{experimental-results}. \sref{conclusion} concludes the paper.
