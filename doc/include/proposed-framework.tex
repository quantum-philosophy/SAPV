In this section, we present a Bayesian framework developed around the proposed idea of the temperature-based UQ of process variation. Let $\mvT_\meas := \vectorize{\{\mT^{(i)}_\meas\}_{i = 1}^\ndata}$ be the vectorization of the temperature measurements $\{ \mT^{(i)}_\meas \}_{i = 1}^\ndata$ in $\Data$ into a single vector with $\ndps = \ndata \nprocs \nsteps$ entries. For convenience, denote the models introduced in \sref{power-model} and \sref{thermal-model} by
\begin{equation} \elabel{model}
  \mvT = \model{\u},
\end{equation}
which, for an outcome $\u$, transforms the test dynamic power profile $\profilePdyn$ into the temperature vector $\mvT := \vectorize{\{\mT^{(i)}\}_{i = 1}^\ndata}$ evaluated at the same spatial locations $\vr$ as the vector $\mvT_\meas$ defined earlier. In what follows, we shall refer to this joint model as the forward model.

\subsection{Statistical Model} \slabel{statistical-model}
Since, at each point of the continuum of spatial locations on the wafer, $\U$ can potentially take a different value, $\U$ is infinite-dimensional. We model $\U$ as a square-integrable stochastic process $\U: \omega \times \domain \to \real$ defined over a spacial domain $\domain$ corresponding to the wafer. Denote the forward model developed in \sref{power-model} and \sref{thermal-model} by
\begin{equation} \elabel{model}
  \mT(\r) = \model{\mP_\dyn, \r | \u}
\end{equation}
which, for an outcome $\u: \domain \to \real$ of the process $\U$, transforms an arbitrary dynamic power profile $\profilePdyn$ into the corresponding temperature profile $\profileT$ computed at location $\r \in \domain$. Note, $\model$ also performs thinning of the output temperature to match the partition $\partition{\mT}$ of $\data$. Since $\U(\r)$ is a random element, so $\mT(\r)$ is. Moreover, even if $\U(\r)$ was given (fixed), due to the imperfection of the modeling and measurement processes, a temperature profile $\mT^{(i)}_\meas$ from the data set $\data$ is assumed to deviate from the model prediction in \eref{model}. To account for this, $\mT^{(i)}_\meas$ is modeled as
\[
  \mT^{(i)}_\meas = \mT(\r_i) + \mnoise^{(i)} = \model{\mP_\dyn, \r_i | \u} + \mnoise^{(i)}
\]
where $\mnoise^{(i)} = (\noise^{(i)}_{jk}) \in \real^{\ncores \times \nsteps}$ represents a matrix of an additive noise. Such a noise is typically assumed to be a white Gaussian noise and to be independent from the parametrization $\U(\r)$ \cite{marzouk2007, el-moselhy2012}. Therefore, $\noise^{(i)}_{jk}$ is modeled as
\[
  \noise^{(i)}_{jk} \sim \gaussian{0}{\sigma^2_\noise}
\]
where, imposing no loss of generality, the noise is assumed to have the same magnitude for all measurements. Now, we can form the likelihood function $\f{\data | \u}$ of the data $\data$, which, intuitively speaking, gives the probability of observing $\data$:
\begin{align}
  \f{\data | \u} & = \prod_{i = 1}^{\ndata} \f{\mT^{(i)}_\meas, \r_i | \u} \nonumber \\
  & = \left(2 \pi \sigma_\noise^2\right)^{-\frac{\ndcs}{2}} \exp\left( -\frac{1}{2 \sigma^2_\noise} \sum_{i = 1}^\ndata \fnorm{\mnoise^{(i)}}^2 \right) \elabel{likelihood}
\end{align}
where $\ndcs = \ndata \ncores \nsteps$, $\fnorm{\cdot}$ is the Frobenius norm, and
\[
  \mnoise^{(i)} = \mT^{(i)}_{\meas} - \mT(\r_i) = \mT^{(i)}_{\meas} - \model{\mP_\dyn, \r_i | \u}.
\]
The likelihood function in \eref{likelihood} corresponds to the density of a Gaussian vector with $\ndcs$ independent components:
\[
  \data | \u \sim \gaussian{\vectorize{\model{\mP_\dyn, \vr | \u}}}{\sigma_\noise^2 \mI_\ndcs}
\]
where $\vectorize{\cdot}$ stands for the vectorization of a matrix or a set of matrices, \ie, columns of matrices are successively stacked into a single vector. In the notation above, the left-hand side is a shortcut for $\vectorize{\{ \mT_\meas^{(i)} \}_{i = 1}^\ndata} | \u$, and the right-hand side means that the forward model $\model$ is to be evaluated for each $\r_i$ in $\vr$, and the resulting matrices are then vectorized.

Now, we need to identify the stochastic process $\U(\r)$ in order to specify the prior, which the density $\f{\u}$ in \eref{bayes} corresponds to. Uncertainties due to process variation are known to be well approximated using Gaussian distributions \cite{srivastava2010}; therefore, $\U(\r)$ is assumed to be a Gaussian process \cite{rasmussen2006}:
\[
  \U(\r) \sim \gaussian{\fMean{\r}}{\fCov{\r, \r'}}
\]
where $\fMean{\r} := \E{\U(\r)}$ and $\fCov{\r, \r'}$ are the expectation and covariance functions of $\U(\r)$. Due to the particularities of the manufacturing process, such uncertainties often bear radial structures \cite{cheng2011}; thus, we assume that $\r$ represents the Euclidean distance from the center of the wafer to the center of the die, and the covariance function of $\U(\r)$ belongs to the squared exponential family of covariance functions \cite{rasmussen2006}:
\begin{equation} \elabel{covariance-function}
  \fCov{\r, \r'} = \sigma^2_\U \exp\left(-\frac{1}{2 \ell}(\r - \r')^2\right)
\end{equation}
where $\sigma^2_\U$ stands for variance, and $\ell > 0$ is the correlation length. Now, let $\vr^* = (\r^*_i) \in \real^n$ be a vector of spatial locations of interest. Then the prior over these locations is
\begin{equation} \elabel{prior}
  \f{\u | \vr^*} = \left(|2 \pi \mCov|\right)^{-\frac{1}{2}} \exp\left(-\frac{1}{2} (\vu - \vfMean)^T \mCov^{-1} (\vu - \vfMean) \right)
\end{equation}
where $\vu = \u(\vr^*) := (u(\r^*_i)) \in \real^n$, $\vfMean = \fMean{\vr^*} := (\fMean{\r^*_i}) \in \real^n$, $\mCov = (\fCov{\r^*_i, \r^*_j}) \in \real^{n \times n}$, and $|\cdot|$, applied to a matrix, denotes the matrix determinant. Equivalently,
\[
  \u | \vr^* \sim \gaussian{\vfMean}{\mCov}
\]
where $\u | \vr^*$ stands for $\u(\vr^*)$.

Due to the forward model $\model$ involved in \eref{likelihood}, there is no convenient expression for the posterior density $\f{\u | \data}$ in \eref{bayes}. As mentioned in \sref{bayesian-inference}, to overcome the difficulty, we utilize the Metropolis-Hastings algorithm \cite{gelman2004}. To this end, we combine \eref{likelihood} with \eref{prior} and compute, up to a constant summand, the logarithm of the posterior:
\begin{align*}
  & \log \f{\u | \data, \vr^*} = \log \f{\data | \u} + \log \f{\u | \vr} + c_1 \\
  & = -\frac{1}{2 \sigma^2_\noise} \sum_{i = 1}^\ndata \fnorm{\mnoise^{(i)}}^2 -\frac{1}{2} (\vu - \vfMean)^T \mCov^{-1} (\vu - \vfMean) + c_2 \\
  & = - A(\u | \data) - B(\u | \vr^*) + c_2
\end{align*}
where $c_1$ and $c_2$ are constants with respect to $\u$, which are irrelevant for our purposes, and
\begin{align*}
  & A(\u | \data ) = \frac{1}{2 \sigma^2_\noise} \sum_{i = 1}^\ndata \fnorm{\mT^{(i)}_{\meas} - \model{\mP_\dyn, \r_i | \u}}^2, \\
  & B(\u | \vr^*) = \frac{1}{2} (\vu - \vfMean)^T \mCov^{-1} (\vu - \vfMean).
\end{align*}
It is worth being noted that $A(\u | \data)$ poses a significant computational challenge as it requires $\ndata$ evaluations of $\model$ for each sample of $\U$.


\subsection{Post-processing} \slabel{post-processing}
At \stage{4}\ in \fref{algorithm}, using the set of samples $\UData$, the user computes the desired statistics of the \qoi\ such as the most probable value of the effective channel length at some location of interest, the probability of a certain area on the wafer to be defective, \etc\ The computations boil down to the estimation of expected values with respect to the posterior distribution of $\vparam$, $\f{\vparam | \QData}$.
This estimation is done in the standard sample-based fashion, that is, in order to compute some arbitrary quantity dependent on $\u$, one needs to evaluate this quantity for each $\vu_i$ in $\UData$ and then take the average.

The strength of the Bayesian approach to inference really starts to shine when we are also interested in assessing the trustworthiness of the measured data and, therefore, the reliability of the estimates/decisions based on these data.
Such an assessment can readily be undertaken using our framework since the delivered posterior distribution contains all the needed information about the \qoi.
This is especially helpful in decision making as exemplified in \sref{motivation}.


\subsection{Computational Aspects} \slabel{computational-aspects}
In this section, we discuss several aspects of our framework that significantly speed up the inference procedure.

\subsection{Thermal Model}
First of all, we need to address the complexity of the forward model $\model$. The model is expensive as it involves the system of nonlinear differential equations given in \eref{heat-de}, which, in general, should be solved numerically using, \eg, Runge-Kutta methods \cite{press2007}. In order to mitigate these repetitive computations, we utilize the approach that we discuss in our earlier publication \cite{ukhov2012} in detail. The idea is that, if the power term on the right-hand side of \eref{heat-de} was constant, the system would be linear and would have an analytical solution, which is much more efficient to work with. Therefore, we assume that the total dissipation of power between two successive moments of time of the input time partition $\partition{\mP}$ (see \sref{problem-formulation}) stays constant. Consequently, we can stride in time solving \eref{heat-de} for each step analytically. See \cite{ukhov2012} for further details.

\subsection{Proposal Distribution} \slabel{proposal-distribution}
We construct a good proposal distribution for the Metropolis algorithm such to the corresponding Markov chains explore the probability space more efficiently. ``Efficiency'' in this case means that one needs to collect much fewer samples, \ie, to perform much fewer forward model evaluations, to draw accurate conclusions from the data. To this end, before the actual sampling process, we optimize the log-posterior function given by \eref{log-posterior} using the Quasi-Newton algorithm \cite{press2007}. The result of this optimization is a posterior mode $\hat{\vparam}$ and the corresponding observed information matrix $\mOI$ \cite{gelman2004}, which form a solid base for the further sampling.

\subsection{Sampling Strategy}
We make use of the omnipresent parallel computing. Therefore, instead of utilizing the classical Metropolis algorithm, which is sequential, we appeal to the independence sampler Metropolis algorithm \cite{gelman2004}. The proposal distribution is chosen to be a multivariate t-distribution:
\[
  \vparam \sim \studentst{\nu}{\hat{\vparam}}{\alpha \mOI}
\]
where $\hat{\vparam}$ and $\mOI$ are as in \sref{proposal-distribution}, $\nu$ is the number of degrees of freedom, and $\alpha > 0$ is a tuning constant. It can be seen that the proposal is independent of the current position of the chain and, therefore, allows one for parallel computations of the posterior in \eref{log-posterior}.

