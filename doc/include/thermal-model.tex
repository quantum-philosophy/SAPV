Given the thermal specification $\specification$ (see \sref{problem-formulation}) of the multiprocessor system at hand, an equivalent representation, capturing the thermal behavior of the system, is constructed \cite{kreith2000}. This representation is known as a thermal RC circuit, which is composed of a number of thermal nodes.
The structure of the circuit depends on the intended level of granularity and, therefore, impacts the resulting accuracy. For clarity of presentation, we assume that each processing element is mapped onto one corresponding node, and the thermal package is represented as a set of additional nodes.
The temperature of the multiprocessor platform is then modeled with the following system of differential-algebraic equations:
\begin{subnumcases}{\elabel{thermal-system}}
  \mC \: \frac{d\vX(\t)}{d\t} + \mG \: \vX(\t) = \mM \: \vP(\t, \vT(\t), \u) \elabel{heat-de} \\
  \vT(\t) = \mM^T \vX(\t) + \vT_\amb \elabel{temperature-output}
\end{subnumcases}
where $\nnodes$ is the number of thermal nodes in the circuit; $\mC \in \real^{\nnodes \times \nnodes}$ is a diagonal matrix of the thermal capacitance; $\mG \in \real^{\nnodes \times \nnodes}$ is a symmetric, positive-definite matrix of the thermal conductance; $\vX \in \real^\nnodes$ is the internal state vector; $\vP \in \real^\nprocs$ and $\mM \in \real^{\nnodes \times \nprocs}$ are the input power vector and its mapping matrix to the thermal nodes; $\vT \in \real^\nprocs$ is the temperature vector, and $\vT_\amb \in \real^\nprocs$ is the vector of the ambient temperature.
In general, \eref{heat-de} does not have a closed-form solution due to the power term defined in \eref{total-power}; hence, the system is typically being solved numerically.

Given a dynamic power profile $\profilePdyn$ and using \eref{total-power} and \eref{thermal-system}, we can now compute the corresponding temperature profile $(\mT, \partition{\mP})$ of one die, where $\mT = (\vT(\t_i))$, $\t_i \in \partition{\mP}$.
Let $\mvT_\meas := \vectorize{\{\mT^{(i)}_\meas\}_{i = 1}^\ndata}$ be the vectorization of the temperature measurements $\{ \mT^{(i)}_\meas \}_{i = 1}^\ndata$ in $\Data$ into a single vector with $\ndps = \ndata \nprocs \nsteps$ entries.
Denote by
\begin{equation} \elabel{model}
  \mvT = \model{\u}
\end{equation}
the model developed so far, which, for an outcome $\u$, transforms the test dynamic power profile $\profilePdyn$ into the temperature vector $\mvT := \vectorize{\{\mT^{(i)}\}_{i = 1}^\ndata}$ evaluated at the same spatial locations $\vr$ as $\mvT_\meas$ defined earlier. In what follows, we shall refer to $\model$ as the forward model.
