Given a thermal specification $\specification$ (see \sref{problem-formulation}), an equivalent thermal RC circuit with $\nnodes$ thermal nodes is constructed \cite{kreith2000}. The structure of the circuit depends on the intended level of granularity and, therefore, impacts the resulting accuracy. For clarity of presentation, we assume that each processing element is mapped onto one corresponding node, and the thermal package is represented as a set of additional nodes.

The thermal behavior of the circuit is modeled with the following system of differential-algebraic equations:
\begin{subnumcases}{\elabel{thermal-system}}
  \mC \: \frac{d\vx(\t)}{d\t} + \mG \: \vx(\t) = \mM \: \vP(\t, \vT(\t)) \elabel{heat-ode} \\
  \vT(\t) = \mM^T \vx(\t) + \vT_\amb \elabel{temperature-output}
\end{subnumcases}
where $\mC \in \real^{\nnodes \times \nnodes}$ is a diagonal matrix of the thermal capacitance; $\mG \in \real^{\nnodes \times \nnodes}$ is a symmetric, positive-definite matrix of the thermal conductance; $\vx \in \real^\nnodes$ is the internal state vector; $\vP \in \real^\nprocs$ and $\mM \in \real^{\nnodes \times \nprocs}$ are the input power vector and its mapping matrix to the thermal nodes; $\vT \in \real^\nprocs$ is the temperature vector, and $\vT_\amb \in \real^\nprocs$ is the vector of the ambient temperature. In general, \eref{heat-ode} does not have a closed-form solution due to the power term defined in \eref{total-power}; hence, the system is typically being solved numerically. Consequently, given a dynamic power profile $\profilePdyn$ and using \eref{total-power} and \eref{thermal-system}, we can compute the corresponding temperature profile $(\mT, \partition{\mP})$ where $\mT = (\vT(\t_i))$, $\t_i \in \partition{\mP}$.

Let $\mvT_\meas := \vectorize{\{\mT^{(i)}_\meas\}_{i = 1}^\ndata}$ be the vectorization of the temperature measurements $\{ \mT^{(i)}_\meas \}_{i = 1}^\ndata$ in $\data$ into a single vector with $\ndps = \ndata \nprocs \nsteps$ entries. For convenience, denote the forward model developed so far by
\begin{equation} \elabel{model}
  \mvT = \model{\u},
\end{equation}
which, for an outcome $\u$, transforms the test dynamic power profile $\profilePdyn$ into the temperature vector $\mvT := \vectorize{\{\mT^{(i)}\}_{i = 1}^\ndata}$ evaluated at the same spatial locations $\vr$ as the vector $\mvT_\meas$ defined earlier.
