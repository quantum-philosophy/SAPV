\subsection{Power Model}
The total dissipation of power is composed of the dynamic power and leakage power. The rest is as usual.

\subsection{Thermal Model}
Based on the thermal specification $\specification$, an equivalent thermal RC circuit of the multiprocessor system under consideration is constructed. The rest is as usual.

\subsection{Statistical Model}
Since, at each point of the continuum of spatial locations on the wafer, the \rv\ $\U$ can potentially take a different value, \eg, the effective channel length of devices are known to vary across the wafer \cite{cheng2011}, $\U$ is infinite-dimensional. Thus, we model $\U$ as a stochastic process $\U(\r)$ indexed by the location parameter $\r$. Denote the forward model developed so far by
\begin{equation} \elabel{model}
  \mT(\r) = \model{\mP_\dyn | \u(\r)},
\end{equation}
which, for a given outcome $\u(\r)$ of the parameters $\U(\r)$, transforms an arbitrary dynamic power profile $\profilePdyn$ to the corresponding temperature profile $\profileT$. Note, the model also performs thinning of the output temperature to match the partition $\partition{\mT}$ of $\data$. Since $\U(\r)$ is a random element, so $\mT(\r)$ is. Moreover, even if $\U(\r)$ was given, due to the imperfection of the modeling and measurement processes, a temperature profile $\mT^{(i)}_\meas$ from the data set $\data$ is assumed to deviate from the model prediction in \eref{model}. To account for this, $\mT^{(i)}_\meas$ is modeled as
\[
  \mT^{(i)}_\meas = \mT(\r_i) + \mnoise^{(i)} = \model{\mP_\dyn | \u(\r_i)} + \mnoise^{(i)}
\]
where $\mnoise^{(i)} = (\vnoise^{(i)}_{jk}) \in \real^{\ncores \times \nsteps}$ represents a matrix of an additive noise. Such a noise is typically assumed to be a white Gaussian noise and to be independent from the parametrization $\U(\r)$ \cite{marzouk2007, el-moselhy2012}. Therefore, $\noise^{(i)}_{jk}$ is modeled as
\[
  \noise^{(i)}_{jk} \sim \gaussian{0}{\sigma^2_\noise}
\]
where, imposing no loss of generality, the noise is assumed to have the same magnitude for all measurements. Now, we can form the likelihood function $\f{\data | \u}$ of the data $\data$, which, intuitively speaking, gives the probability of observing $\data$:
\begin{align*}
  & \f{\data | \u} = \prod_{i = 1}^{\ndata} \prod_{j = 1}^{\ncores} \prod_{k = 1}^{\nsteps} \f{\mT^{(i)}_{\meas, jk} | \u, \r_i} \\
  & = \prod_{i = 1}^{\ndata} \prod_{j = 1}^{\ncores} \prod_{k = 1}^{\nsteps} \frac{1}{\sqrt{2 \pi} \sigma_\noise} \exp\left( -\frac{1}{2 \sigma^2_\noise} \left(\noise_{jk}^{(i)}\right)^2 \right) \\
  & = \frac{1}{(\sqrt{2 \pi} \sigma_\noise)^{\ndata \ncores \nsteps}} \exp\left( -\frac{1}{2 \sigma^2_\noise} \sum_{i = 1}^{\ndata} \sum_{j = 1}^{\ncores} \sum_{k = 1}^{\nsteps} \left(\noise_{jk}^{(i)}\right)^2 \right)
\end{align*}
where $\noise^{(i)}_{jk}$ are the elements of the matrix
\[
  \vnoise^{(i)} = \mT^{(i)}_{\meas} - \mT(\r_i) = \mT^{(i)}_{\meas} - \model{\mP_\dyn | \u(\r_i)}.
\]
It can be seen that the forward model should be evaluated as many times as many observations we have, \ie, $\ndata$ times.

Now, we need to identify the stochastic process $\U(\r)$ in order to specify the prior distribution, which the density $\f{\u}$ in \eref{bayes} corresponds to. Uncertainties due to process variation are known to be well approximated using Gaussian distributions \cite{srivastava2010}; therefore, $\U(\r)$ is assumed to be a Gaussian process \cite{rasmussen2006}:
\[
  \U(\r) \sim \gaussianp{\fMean{\r}}{\fCov{\r, \r'}}
\]
where $\fMean{\r} = \E{\U(\r)}$ and $\fCov{\r, \r'} = \Cov{\U(\r), \U(\r')}$ are the expectation and covariance functions of $\U(\r)$. Due to the particularities of the manufacturing process, such uncertainties often bear radial structures \cite{cheng2011}; thus, we assume that $\r$ represents the Euclidean distance from the center of the wafer to the center of the die, and the covariance function of $\U(\r)$ is given as follows \cite{maitre2010}:
\begin{equation} \elabel{covariance-function}
  \fCov{\r, \r'} = \sigma^2_\U \exp\left(-\frac{|\r - \r'|}{\eta}\right)
\end{equation}
where $\sigma^2_\U$ stands for variance, and $\eta$ is the correlation length. Now, let $\r = (\r_i) \in \real^n$ be a vector of spatial locations that we are interested in. Then the prior over these locations is
\[
  \f{\u, \r} = \frac{1}{\sqrt{|2 \pi \mCov|}} \exp\left(-\frac{s^T \mCov^{-1} s}{2} \right)
\]
where $s = (s_i) \in \real^n$ is a vector computed as $s_i = \u(\r_i) - \fMean{\r_i}$, $\mCov = (\mCov_{ij}) \in \real^{n \times n}$ is a covariance matrix defined as $\mCov_{ij} = \fCov{\r_i, \r_j}$, and $|\cdot|$, applied to a matrix, denotes the matrix determinant.
