Having completed the MC sampling procedure, we obtain a data set of $\nsamples$ samples of $\u$ denoted by $\UData = \{ \u_i \}_{i = 1}^\nsamples$. Now, we can compute a wide range of statistics from $\UData$, both across the wafer and across a single die: moments, probabilities, quantiles, \etc\ It is worth being noted that such statistics can be estimated not only for $\u$ itself, but also for other quantities dependent on $\u$. For example, we can revert the inference and, given an arbitrary dynamic power profile (different from the one used to collect $\Data$), reason about the corresponding power and temperature profiles, \eg, find the probability density function of the maximal values.

The strength of the Bayesian approach to inference starts really shine when one needs to take a decision of some kind based on the collected observations in $\Data$. An example of such a decision is given in \sref{motivation}.
