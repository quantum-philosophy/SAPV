Uncertainty quantification (UQ) is an essential component of the multiprocessor system design workflow. In this paper, we address an inverse UQ problem targeted at inferring the parameters that the power dissipation and, consequently, temperature of multiprocessor systems depend on. Specifically, we focus on those parameters that are severely affected by process variation, \eg, the effective channel length, gate oxide thickness, threshold voltage, \etc\ Process variation renders such quantities of interest as being intrinsically uncertain; in other words, they are random variables (\rvs).

The proposed UQ framework is based on Bayesian inference. In general, such an approach relies on Monte Carlo (MC) sampling techniques, \eg, Markov chain Monte Carlo (MCMC). Even though MC sampling is flexible in tackling diverse problems, it is dramatically slow as it requires repetitive evaluations of the forward model. In our scenario, the forward model is a system of differential equations, with no closed-form solution, describing heat transfer within a multiprocessor platform; consequently, high computational costs follow. To circumvent the difficulty, we substitute the heavy forward model with a light surrogate model. For this purpose, we utilize the Gaussian process (GP) regression \cite{rasmussen2006}; however, the proposed framework is not limited to a particular choice.

In the context of multiprocessor systems, inverse problems have drawn a lot of attention. Bayesian inference is utilized in \cite{zhang2010} to identify an optimal set of locations on the wafer wherein measurements should be taken in order to quantify the uncertainties due to the manufacturing process with the maximal accuracy. In \cite{paek2012}, the authors address the inference of the power dissipation based on transient temperature profiles using Markov random fields. A temperature-based characterization of the parameters affecting power is also considered in \cite{mesa-martinez2007} wherein a genetic algorithm is employed for model fitting.

The reminder of this paper is organized as follows. In \sref{problem-formulation}, we formulate the objective of our study. Preliminary materials are given in \sref{preliminaries}. The proposed UQ framework is presented in \sref{proposed-framework}. Experimental results are reported in \sref{experimental-results}. \sref{conclusion} concludes the paper.
