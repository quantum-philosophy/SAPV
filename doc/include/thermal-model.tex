Using the layout-and-temperature-related information given in $\specification$ (see \sref{problem-formulation}), we construct a so-called thermal RC circuit of the system that captures the thermal behavior of a single die on the wafer \cite{kreith2000}.
The circuit is composed of $\nnodes$ thermal nodes, and its structure depends on the intended level of granularity and, therefore, impacts the resulting accuracy. For clarity, we assume that each processing element is mapped onto one thermal node, and there can be other nodes corresponding to the passive elements of the die.
The transient temperature of a die is then modeled using the following system of differential-algebraic equations:
\begin{subnumcases}{\elabel{thermal-system}}
  \mC \: \frac{d\vX(\t)}{d\t} + \mG \: \vX(\t) = \mM \: \vP(\t, \vT(\t), \u) \elabel{heat-de} \\
  \vT(\t) = \mM^T \vX(\t) + \vT_\amb \elabel{temperature-output}
\end{subnumcases}
where $\nnodes$ is the number of thermal nodes in the circuit; $\mC \in \real^{\nnodes \times \nnodes}$ is a diagonal matrix of the thermal capacitance; $\mG \in \real^{\nnodes \times \nnodes}$ is a symmetric, positive-definite matrix of the thermal conductance; $\vX(\t) \in \real^\nnodes$ is the state vector; $\vP(\t) \in \real^\nprocs$ and $\mM \in \real^{\nnodes \times \nprocs}$ are the input power vector and its mapping matrix to the thermal nodes; $\vT(\t) \in \real^\nprocs$ is the output temperature vector, and $\vT_\amb \in \real^\nprocs$ is the vector of the ambient temperature.
Given a dynamic power profile $\profilePdyn$ and using \eref{total-power} and \eref{thermal-system}, we can now compute the corresponding temperature profile $(\mT, \partition{\mP})$ of one die.
At this point, the obtained temperature profile contains data for all $\nprocs$ processing elements and for all $\nsteps$ moments of time dictated by $\profilePdyn$.
