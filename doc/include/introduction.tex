Process variation constitutes one of the major concerns of electronic system designs \cite{chandrakasan2001, srivastava2010}. A crucial implication of process variation is that it renders the key parameters of a technological process, \eg, the effective channel length and gate oxide thickness, as uncertain quantities.
Therefore, the same workload applied to two ``identical'' dies can lead to two different power and, thus, temperature profiles since the dissipation of power and heat essentially depends on the aforementioned stochastic quantities.
Consequently, process variation leads to performance degradation in the best case and to severe faults or burnt silicon in the worst scenario.
Under these circumstances, uncertainty quantification has evolved into an indispensable asset of the fabrication workflows that can provide high guaranties on the efficiency and robustness of their products.

An important target of uncertainty quantification is the characterization of the on-wafer distribution of a quantity of interest (\qoi), deteriorated by process variation, based on a data set of measurements.
The problem belongs to the class of so-called inverse problems since the \qoi\ can be seen as an input to the system and the measurements as the corresponding output, which is opposed to direct problems wherein one studies outputs based on the knowledge of inputs (see, \eg, \cite{juan2011, juan2012}).
Such an inverse problem is addressed in this paper: our goal is to quantify distributions of the key process parameters, \eg, the effective channel length, and we approach this goal by measuring transient temperature profiles and analyzing them with the help of Bayesian inference \cite{gelman2004}.

Bayesian inference is utilized in \cite{zhang2010} to identify the optimal set of locations on a wafer, in which the \qois\ should be measured in order to characterize them with the maximal accuracy.
In \cite{paek2012}, the authors consider an inverse problem focused on the inference of the power dissipation based on transient temperature maps using Markov random fields.
Another temperature-based characterization of power is developed in \cite{mesa-martinez2007} wherein a genetic algorithm is employed for the reconstruction of the power model.
It should be noted that the approach in \cite{zhang2010} requires test structures to be deployed onto each die on the wafer as it operates on direct measurements, which can be expensive and, thus, impractical to undertake.
Moreover, the technique in \cite{zhang2010} analyzes such quantities as frequencies, voltages, and currents, but not the primary sources of uncertainty such as the effective channel length and gate oxide thickness.
The approaches in \cite{paek2012} and \cite{mesa-martinez2007}, on the other hand, concentrating on the power dissipation of a single die, are not concerned with process variation.

Our work makes the following main contributions. First, we propose a novel approach to the quantification of process variation based on indirect, incomplete, and noisy measurements. Indirectness is the key ingredient as it allows for a significant decrease of the costs associated with the process-variation characterization.
Second, we develop and implement a solid framework around the proposed idea. Third, we perform a thorough study of various aspects of the framework that are of high practical interests.
