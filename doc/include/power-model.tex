The total dissipation of power of a system with $\nprocs$ processing elements at time $\t$ is composed of the dynamic and leakage parts:
\begin{equation} \elabel{total-power}
  \vP(\t, \vT(\t), \u) = \vP_\dyn(\t) + \vP_\leak(\vT(\t), \u)
\end{equation}
where $\vP, \vT \in \real^\nprocs$ are vectors of power and temperature, respectively, and $\u$ is an outcome of the uncertain parametrization. The dynamic part, $\vP_\dyn$, is temperature-independent whereas the leakage part, $\vP_\leak$, is a strong function of the operating temperature.
The influence of process variation on the dynamic power is known to be negligibly small \cite{srivastava2010, juan2011, juan2012}; therefore, only leakage is assumed to be dependent on $\u$.
The choice of the leakage model is irrelevant for the present paper and, therefore, will not be discussed any further; refer to the above-mentioned literature for concrete examples.
