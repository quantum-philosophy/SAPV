In this section, we present a Bayesian framework developed around the proposed idea of the temperature-based UQ of process variation. Let $\mvT_\meas := \vectorize{\{\mT^{(i)}_\meas\}_{i = 1}^\ndata}$ be the vectorization of the temperature measurements $\{ \mT^{(i)}_\meas \}_{i = 1}^\ndata$ in $\Data$ into a single vector with $\ndps = \ndata \nprocs \nsteps$ entries. For convenience, denote the models introduced in \sref{power-model} and \sref{thermal-model} by
\begin{equation} \elabel{model}
  \mvT = \model{\u},
\end{equation}
which, for an outcome $\u$, transforms the test dynamic power profile $\profilePdyn$ into the temperature vector $\mvT := \vectorize{\{\mT^{(i)}\}_{i = 1}^\ndata}$ evaluated at the same spatial locations $\vr$ as the vector $\mvT_\meas$ defined earlier. In what follows, we shall refer to this joint model as the forward model.

\subsection{Statistical Model} \slabel{statistical-model}
Since, at each point of the continuum of spatial locations on the wafer, the random element $\u$ can potentially take a different value, $\u$ is infinite-dimensional.
We model $\u$ as a square-integrable stochastic process $\u: \outcomes \times \domain \to \real$ defined over a spatial domain $\domain$ (see also \aref{kl-expansion}), which corresponds to the wafer.
Uncertainties due to process variation are known to be well approximated using Gaussian distributions \cite{srivastava2010}; therefore, $\u$ is assumed to be a Gaussian process \cite{rasmussen2006}:
\[
  \u | \vparam_\u \sim \gaussianp{\fMean}{\fCov}
\]
where $\fMean$ and $\fCov$ are the mean and covariance functions of $\u$, and $\vparam_\u$ denotes their parametrization. For simplicity, let $\fMean{\r} = \mu$, $\forall \r \in \domain$, meaning that the expected value is constant.
$\fCov$ is chosen to be the following composition:
\begin{equation} \elabel{covariance-function}
  \fCov{\r, \r'} = \sigma_\u^2 \big( \eta \fCov_\SE(\r, \r') + (1 - \eta) \fCov_\OU(\r, \r') \big)
\end{equation}
where
\begin{align*}
  & \fCov_\SE(\r, \r') = \exp\left(-\frac{\norm{\r - \r'}^2}{\ell_\SE^2}\right) \text{ and} \\
  & \fCov_\OU(\r, \r') = \exp\left(- \frac{\abs{\,\norm{\r} - \norm{\r'}\,}}{\ell_\OU} \right)
\end{align*}
are the squared exponential and Ornstein-Uhlenbeck kernels, respectively; $\sigma_\u^2$ represents the variance of $\u$; $\eta \in [0, 1]$ weights the kernels; $\ell_\SE$ and $\ell_\OU > 0$ are the length-scale parameters; $\norm{\cdot}$ stands for the Euclidean distance.
The choice of the covariance function is guided by the observations of the correlation structures induced by the manufacturing process \cite{cheng2011}: the first kernel, $\fCov_\SE$, imposes similarities between the points on the wafer that are close to each other while the second kernel, $\fCov_\OU$, imposes similarities between points that are at the same distance from the center of the wafer.
In this work, $\eta$, $\ell_\SE$, and $\ell_\OU$ are assumed to be given while $\mu$ and $\sigma^2_\u$ are a part of our inference. Thus, we let $\vparam_\u = \{ \mu, \sigma_\u^2 \}$.

\subsubsection{Model order reduction} \slabel{model-order-reduction}
The infinite-dimensional object $\u$ is reduced to a finite-dimensional one via the Karhunen-Lo\`{e}ve (KL) expansion introduced in \aref{kl-expansion}. The discretization is performed with respect to the spatial locations of all $\ncp = \nchips \nprocs$ processing elements on the wafer. Consequently, we obtain an $\ncp$-dimensional \rv\ denoted by $\vu: \outcomes \to \real^\ncp$:
\begin{equation} \elabel{kl-approximation}
  \vu = \mu \vI + \sigma_\u \mKL \vz
\end{equation}
where we treat the constant multiplier $\sigma^2_\u$ in \eref{covariance-function} separately, $\vz = (\z_i) \in \real^\nvars$ obey the standard Gaussian distribution, and $\vI = (e_i = 1)$. Note that model order reduction (see \aref{kl-expansion}) is implied in \eref{kl-approximation}; thus, $\vz \in \real^\nvars$ where $\nvars \leq \ncp$. In addition, we denote by $\vu_\data \in \real^{\ndp}$, $\ndp = \ndata \nprocs$, those elements of $\vu$ that correspond to the observations in $\Data$. Let us redefine $\vparam_\u = \{ \vz, \mu, \sigma^2_\u \}$ and denote the forward model by $\model{\vparam_\u}$.

\subsubsection{The likelihood function}
Due to the imperfection of the measurement process, the temperature profiles in $\Data$, stacked into $\mvT_\meas$, are assumed to deviate from the model prediction in \eref{model}. To account for this,
\[
  \mvT_\meas = \model{\vparam_\u} + \vnoise = \mvT + \vnoise
\]
where $\vnoise$ is an $\ndps$-dimensional vector of noise. The noise is typically assumed to be a white Gaussian noise and to be independent of $\u$ \cite{rasmussen2006, marzouk2009}. Therefore,
\[
  \vnoise | \sigma^2_\noise \sim \gaussian{0}{\sigma^2_\noise \mI}
\]
where $\sigma^2_\noise$ is a parameter defining the variance of the noise; imposing no loss of generality, the noise is assumed to have the same magnitude for all measurements. The measurement noise can be interpreted as
\begin{equation} \elabel{likelihood}
  \mvT_\meas | \vparam_\u, \sigma_\noise^2 \sim \gaussian{\mvT}{\sigma_\noise^2 \mI}
\end{equation}
yielding the likelihood function of the data $\Data$.

\subsubsection{The prior}
Denote the parameters to be inferred as
\[
  \vparam = \vparam_\u \cup \{ \sigma_\noise^2 \} = \{ \vz, \mu, \sigma_\u^2, \sigma_\noise^2 \}.
\]
We put the following priors on $\vparam$:
\begin{align}
  & \z_i \sim \gaussian{0}{1}, \elabel{z-prior} \\
  & \mu \sim \gaussian{\mu_0}{\sigma^2_0}, \elabel{mu-u-prior} \\
  & \sigma^2_\u \sim \sichisquared{\nu_\u}{\tau^2_\u}, \text{ and} \elabel{sigma2-u-prior} \\
  & \sigma^2_\noise \sim \sichisquared{\nu_\noise}{\tau^2_\noise}. \elabel{sigma2-noise-prior}
\end{align}
The prior for $\vz$ is due to the properties of the KL expansion. The next three priors, \ie, a Gaussian and two scaled inverse chi-squared distributions, are a common choice for a Gaussian model with the mean and variance being unknown \cite{gelman2004}. The hyperparameters $\mu_0$, $\tau_\u$, and $\tau_\u$ represent the presumable values of $\mu_u$, $\sigma_\u$, and $\sigma_\noise$, respectively, and the hyperparameters $\sigma_0$, $\nu_\u$, and $\nu_\noise$ reflect the degree of our beliefs according to our prior knowledge. In the absence of such knowledge, non-informative priors can be chosen (see, \eg, \cite{gelman2004}).

\subsubsection{The posterior}
Taking the product of the likelihood in \eref{likelihood} and the priors in \eref{z-prior}--\eref{sigma2-noise-prior}, we obtain
\begin{align}
  & \ln \f{\vparam | \Data} + c = -\frac{\ndps}{2} \ln \sigma^2_\noise - \frac{\norm{\mvT_\meas - \mvT}^2}{2 \sigma^2_\noise} \nonumber \\
  & {} - \frac{\norm{\vz}^2}{2} - \frac{(\mu - \mu_0)^2}{2 \sigma^2_0} - \left(1 + \frac{\nu_\u}{2}\right) \ln \sigma^2_\u - \frac{\nu_\u \tau_\u^2}{2 \sigma^2_\u} \nonumber \\
  & {} - \left(1 + \frac{\nu_\noise}{2}\right) \ln \sigma^2_\noise - \frac{\nu_\noise \tau_\noise^2}{2 \sigma^2_\noise} \elabel{log-posterior}
\end{align}
where $c$ is some constant. This expression is sufficient for the Metropolis algorithm (see \aref{bayesian-inference}); thus, we can readily draw samples from the posterior. Each sample of $\vparam_\u$ is then used in \eref{kl-approximation} to compute a sample of $\u$, \ie, the QoI that we are concerned with, for all processing elements on the wafer.
Note, however, the likelihood function poses a significant computational challenge as each sample requires an evaluation of $\model$; we shall address this issue in \sref{computational-aspects}.


\subsection{Post-processing} \slabel{post-processing}
At Stage~4 in \fref{algorithm}, using the set of samples $\UData$, the user computes the desired statistics about the \qoi\ such as the most probable value of the effective channel length at an arbitrary point on the wafer, the probability of an area on the wafer to be defective, \etc\ The computations boil down to the estimation of expected values with respect to the posterior distribution of $\vparam$. This estimation is done in the standard sample-based fashion, that is, in order to compute the expected value of some quantity $h$, one needs to evaluate $h$ for each $\vu_i$ in $\UData$ and then take the average value:
\[
  \E_{\vparam | \Data}(h(\u)) = \int h(\u) \f{\vparam | \Data} d\vparam \approx \frac{1}{\nsamples} \sum_{i = 1}^\nsamples h(\vu_i).
\]
It is worth emphasizing that such statistics can be estimated not only for $\u$ itself, but also for other quantities dependent on $\u$.
For example, we can revert the inference and, given an arbitrary dynamic power profile (different from the one used to collect $\Data$), reason about the corresponding power and temperature profiles, \eg, find the probability density function of the maximal values.
Also, the strength of the Bayesian approach to inference really starts to shine when one needs to take a decision of some kind based on the collected observations in $\Data$; recall the discussion in \sref{motivation}.


\subsection{Computational Aspects} \slabel{computational-aspects}
In this subsection, we discuss several aspects of our framework that help to speed up the inference procedure.

\subsubsection{Forward Model} \slabel{analytical-solution}
First of all, we address the complexity of $\model$. The model is expensive as it involves a system of nonlinear differential equations (see \eref{heat-de}), which should be solved numerically using, \eg, Runge-Kutta methods \cite{press2007}. In order to mitigate these repetitive computations, we utilize the approach discussed in detail in \cite{ukhov2012}.
The idea is that, if the power term on the right-hand side of \eref{heat-de} stays constant, the system becomes linear and obtains an analytical solution. When the simulated time interval was short enough, the technique was found to have a negligibly small influence on the resulting accuracy; however, the speedup was found to be considerable.
Since $\profilePdyn$ is fine-grained, we can assume that the total power changes only at the time moments $\t_i \in \partition{\mP}$ (see \sref{problem-formulation}). In this way, we can stride in time solving \eref{heat-de} for each step analytically and, thus, gaining a significant speedup. See \cite{ukhov2012} for further details.

\subsubsection{Proposal Distribution} \slabel{proposal-distribution}
The core of the Metropolis algorithm is the proposal distribution. This distribution has a strong impact on the efficiency of the probability space exploration. ``Efficiency'' in this case means that fewer samples, \ie, fewer forward model evaluations, are needed to draw justifiable conclusions from the data.
Therefore, before the actual sampling, we undertake an optimization procedure of the log-posterior function given by \eref{log-posterior}. The result of this optimization is a posterior mode $\hat{\vparam}$ and the corresponding observed information matrix $\mOI$ \cite{gelman2004}, which form a solid base for the proposal distribution.
A popular choice of such a distribution is a multivariate Gaussian distribution wherein the mean is the current location of the chain starting from $\hat{\vparam}$, and the covariance matrix is the inverse of $\mOI$.

\subsubsection{Sampling Strategy} \slabel{sampling-strategy}
We make use of the omnipresent parallel computing for sampling. To this end, instead of utilizing the classical proposal mentioned in \sref{proposal-distribution}---which is purely sequential as the mean for the next sample draw depends on the previous sample---we appeal to the independence sampler Metropolis algorithm \cite{gelman2004}. In this case, a typical choice of the proposal is a multivariate t-distribution, which is independent of the current position of the chain:
\begin{equation} \elabel{proposal}
  \vparam \sim \studentst{\nu}{\hat{\vparam}}{\alpha^2 \mOI^{-1}}
\end{equation}
where $\hat{\vparam}$ and $\mOI$ are as in \sref{proposal-distribution}, $\nu$ is the number of degrees of freedom, and $\alpha$ is a tuning constant. Now, the posterior in \eref{log-posterior} can be computed for all samples in parallel.

