In this section, we describe the objective of Stage~2 in \fref{algorithm}.
Since drawing samples directly from the posterior distribution in \eref{log-posterior} is not possible (see \sref{bayesian-inference}), we utilize the Metropolis-Hastings algorithm \cite{gelman2004}.
The algorithm operates on an auxiliary distribution, called the proposal distribution, which is chosen to be convenient for sampling.
Although the requirements to such a distribution are rather weak, it is often difficult to pick an efficient proposal, which would yield a good approximation to the posterior distribution and, thus, would considerably reduce the number of the forward model evaluations.
This choice is especially severe for high-dimensional problems, and our inference, involving around 30 parameters as we shall see in \sref{experimental-results}, is one of such problems.
Therefore, a careful construction of the proposal distribution is an essential component of our framework.
A common technique to construct a high-quality proposal is to perform an optimization procedure of the log-posterior given by \eref{log-posterior}.
More specifically, we seek for such a value of $\vparam$ that maximizes \eref{log-posterior}; in this paper, this maximization is done via the Quasi-Newton algorithm with BFGS updates of the Hessian matrix \cite{press2007}.
Having done the optimization, we obtain the most probable value of $\vparam$, called the posterior mode and denoted by $\hat{\vparam}$, and also compute the negative of the Hessian matrix at $\hat{\vparam}$, called the observed information matrix and denoted by $\mOI$ (see \fref{algorithm}).
The latter forms a solid base for the proposal as we shall discuss next.
