The total power of a system with $\ncores$ processing elements at time $\t$ is composed of the dynamic and leakage parts:
\begin{equation} \elabel{total-power}
  \vP(\t, \vT(\t), \u) = \vP_\dyn(\t) + \vP_\leak(\vT(\t), \u)
\end{equation}
where $\vP, \vT \in \real^\ncores$ are vectors of power and temperature, respectively, and $\u$ is an outcome of the parametrization $\U$. The dynamic part, $\vP_\dyn$, is temperature-independent whereas the leakage part, $\vP_\leak$, is a strong function of the operating temperature. The influence of process variation on the dynamic power is known to be negligibly small \cite{srivastava2010}; therefore, only leakage is assumed to be dependent on $\U$.


Given the thermal specification $\specification$ (see \sref{problem-formulation}) of the multiprocessor system at hand, an equivalent representation, capturing the thermal behavior of the system, can be constructed \cite{kreith2000}. This representation is known as a thermal RC circuit, which is composed of a number of thermal nodes. The structure of the circuit depends on the intended level of granularity and, therefore, impacts the resulting accuracy. For clarity of presentation, we assume that each processing element is mapped onto one corresponding node, and the thermal package is represented as a set of additional nodes. Temperature of the multiprocessor platform is then modeled with the following system of differential-algebraic equations:
\begin{subnumcases}{\elabel{thermal-system}}
  \mC \: \frac{d\vX(\t)}{d\t} + \mG \: \vX(\t) = \mM \: \vP(\t, \vT(\t), \u) \elabel{heat-de} \\
  \vT(\t) = \mM^T \vX(\t) + \vT_\amb \elabel{temperature-output}
\end{subnumcases}
where $\nnodes$ is the number of thermal nodes in the circuit; $\mC \in \real^{\nnodes \times \nnodes}$ is a diagonal matrix of the thermal capacitance; $\mG \in \real^{\nnodes \times \nnodes}$ is a symmetric, positive-definite matrix of the thermal conductance; $\vX \in \real^\nnodes$ is the internal state vector; $\vP \in \real^\nprocs$ and $\mM \in \real^{\nnodes \times \nprocs}$ are the input power vector and its mapping matrix to the thermal nodes; $\vT \in \real^\nprocs$ is the temperature vector, and $\vT_\amb \in \real^\nprocs$ is the vector of the ambient temperature. In general, \eref{heat-de} does not have a closed-form solution due to the power term defined in \eref{total-power}; hence, the system is typically being solved numerically. Consequently, given a dynamic power profile $\profilePdyn$ and using \eref{total-power} and \eref{thermal-system}, we can compute the corresponding temperature profile $(\mT, \partition{\mP})$ where $\mT = (\vT(\t_i))$, $\t_i \in \partition{\mP}$.


\slabel{analytical-solution}
The thermal model above is expensive as it involves a system of nonlinear differential equations (see \eref{heat-de}), which should be solved numerically using, \eg, Runge-Kutta methods \cite{press2007}. In order to mitigate these repetitive computations, we utilize the approach discussed in \cite{ukhov2012}.
The idea is that, if the power term on the right-hand side of \eref{heat-de} stays constant, the system becomes linear and obtains an analytical solution. When the simulated time interval was short enough, the technique was found to have a negligibly small influence on the resulting accuracy; however, the speedup was found to be considerable.
Since $\profilePdyn$ is fine-grained, we can assume that the total power changes only at the time moments $\t_i \in \partition{\mP}$ (see \sref{problem-formulation}). In this way, we can stride in time solving \eref{heat-de} for each step analytically and, thus, gaining a significant speedup.

Now we apply the derivation above to evaluate $\nrdies$ temperature profiles at the spatial locations of the measurements in $\Data$, \ie, at $\{ \r_i \}_{i = 1}^\nrdies$. The resulting temperature profiles are then shrunk to keep data only for those processing elements and only for those moments of time that are present in the measured temperature profiles, \ie, we keep data only for $\nrprocs$ (out of $\nprocs$) processing elements and only for $\nrsteps$ (out of $\nsteps$) moments of time. Finally, the trimmed profiles are stacked into one vector with $\nrdies \nrprocs \nrsteps$ elements. With a slight abuse of notation, this vector is denoted by $\mvT$ and the overall procedure by $\model$, which we shall refer to as the forward model:
\begin{equation} \elabel{model}
  \mvT = \model{\u}.
\end{equation}
Also, let $\mvT_\meas \in \real^{\nrdies \nrprocs \nrsteps}$ be the stacked version of the measurements in $\Data$, preserving the one-to-one spatial correspondence between the respective elements of $\mvT$ and $\mvT_\meas$.
