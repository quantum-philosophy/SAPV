\subsection{Power Model}
The total dissipation of power is composed of the dynamic power and leakage power.

\subsection{Thermal Model}
Based on the thermal specification $\specification$, an equivalent thermal RC circuit of the multiprocessor system under consideration is constructed.

\subsection{Statistical Model}
Denote the model developed so far by
\begin{equation} \elabel{model}
  \mT = \model{\mP_\dyn | \vU},
\end{equation}
which, for a given outcome of the parameters $\vU$, transforms an arbitrary dynamic power profile $\profilePdyn$ to the corresponding temperature profile $\profileT$. In our settings, $\vU$ is a \rv; therefore, $\mT$ is a \rv\ as well. Moreover, due to the imperfection of the modeling and measurement processes, a temperature profile $\mT^{(i)}_\meas$ from the given data set $\data$ is assumed to deviate from the model prediction in \eref{model} even if $\vU$ was deterministic. Consequently, $\mT^{(i)}_\meas$ is modeled as
\[
  \mT^{(i)}_\meas = \mT^{(i)} + \mnoise^{(i)} = \model{\mP_\dyn | \vU^{(i)}} + \mnoise^{(i)}
\]
where $\vU^{(i)}$ is a copy of $\vU$ that belongs to the $i$th die, and $\mnoise^{(i)} = ( \vnoise^{(i)}_1, \dotsc, \vnoise^{(i)}_{\nsteps} ) \in \real^{\ncores \times \nsteps}$ represents a matrix of an additive noise. Such a noise is typically assumed to be a Gaussian white noise and to be independent from the parametrization $\vU^{(i)}$, $\forall i$ \cite{marzouk2007, el-moselhy2012}. Therefore, each column vector $\vnoise^{(i)}_j$, $j = 1, \dotsc, \nsteps$, of $\mnoise^{(i)}$ is modeled as
\[
  \vnoise^{(i)}_j \sim \normal{\vZero}{\mSigma^{(i)}_j}
\]
where $\mSigma^{(i)}_j \in \real^{\ncores \times \ncores}$ is a covariance matrix.
