In order to give a clear presentation of the proposed technique, we first overview some of the major components of Bayesian inference \cite{gelman2004}.
Let $\vparam$ be a set of unknown for us parameters, which we would like to characterize. Our arsenal to combat the problem includes: (a) a data set of observations $\Data$, (b) a model $\model$ that connects $\vparam$ with $\Data$, and (c) our prior knowledge on $\vparam$. A natural solution is widely celebrated Bayes' rule:
\[
  \f{\vparam | \Data} \propto \f{\Data | \vparam} \f{\vparam}
\]
where $\f{\cdot}$ denotes a probability density function, and the vertical bar stands for conditioning \cite{durrett2010}.
$\f{\Data | \vparam}$ is known as the likelihood function, which accommodates the model $\model$ and yields the probability of observing the data $\Data$ given the parameters $\vparam$.
$\f{\vparam}$ is called the prior of $\vparam$, which represents our knowledge on $\vparam$ prior to any observations.
Finally, $\f{\vparam | \Data}$ reads as the posterior of $\vparam$ given $\Data$. Such a posterior is an exhaustive solution to our problem: having constructed $\f{\vparam | \Data}$, one can perform a thorough analysis of $\u$ by drawing samples from this posterior and then estimating various quantities.

Unfortunately, due to $\model$ involved in the likelihood function, the posterior is often not available in closed form and, therefore, the sampling procedure is typically not straightforward.
To tackle the difficulty, one usually relies on such techniques as Markov Chain Monte Carlo (MCMC) sampling \cite{gelman2004}. In this case, an ergodic Markov chain with the stationary distribution equal to the target posterior distribution is constructed and then utilized for the probability space exploration.
A popular instantiation of MCMC sampling is the Metropolis algorithm wherein such a Markov chain is attained via sampling from a so-called proposal distribution, which is chosen to be computationally convenient; each such sample is then accepted or rejected according to a certain strategy. This acceptance/rejection strategy forces the resulting chain of samples to move towards regions of high probability and, after a sufficient number of steps, to produce samples from the desired posterior distribution. The first portion of the drawn samples is typically discarded before the final computations as being unrepresentative; this portion is also known as the burn-in period. It should be noted that the proposal distribution has a dramatic influence of the overall sampling procedure as a wisely chosen proposal allows for a more rapid exploration of the probability space wherein fewer samples are needed to draw justifiable conclusions.
